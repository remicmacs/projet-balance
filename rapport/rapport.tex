\documentclass[a4paper,11pt,titlepage]{article}
\usepackage[T1]{fontenc}
\usepackage[utf8]{inputenc}
\usepackage{lmodern}
\usepackage[french]{babel}
\usepackage{graphicx}
\usepackage{color}
\usepackage{hyperref}
\usepackage{listings}
\usepackage{pgfplots}
\usepackage{tabularx}
\usepackage{xcolor}

\newcolumntype{C}{>{\centering\arraybackslash}X}

\hypersetup{
    colorlinks=true, % make the links colored
    linkcolor=blue, % color TOC links in blue
    urlcolor=gray, % color URLs in red
    linktoc=all % 'all' will create links for everything in the TOC
}

\lstset{
  basicstyle=\ttfamily,
  captionpos=b,
  language=c,
  frame=single,
  breaklines=true,
  upquote=true,
  commentstyle=\color{mygreen},
  prebreak=\space\textbackslash
}

\title{Rapport projet électronique}
\author{Rémi BOURGEON\\
\texttt{remi.bourgeon@isen.yncrea.fr}\\\\
Rodolphe HOUDAS\\
\texttt{rodolphe.houdas@isen.yncrea.fr}}
\date{2017-2018}

\begin{document}

\maketitle
\tableofcontents
\newpage

%\begin{abstract}
%\end{abstract}

\section{Description du projet}

\section{Bits de configuration}

Conformément a ce qui avait été fait lors des TP d'électronique numérique, nous avons changé les bits de configuration suivants:\\

\begin{lstlisting}
FOSC  INTIO7
WDTEN OFF
LVP   OFF
\end{lstlisting}

Nous réglons l'horloge à 4Mhz à l'aide du registre \texttt{OSCCON} :\\

% Ok ici on a un bon tableau
% Le noindent est pour enlever l'indentation (sinon le tableau déborde dans l'autre sens
% Ici on utilise tabularx qui est une version améliorée de tabular
% |*{8}{C|} signifie qu'on crée 8 colonnes avec un bord (d'où le |) et dont le contenu est centré
\noindent
\begin{tabularx}{\textwidth}{|c|C|C|C|c|c|C|C|}
  \hline
  \multicolumn{8}{|>{\hsize=8\hsize}c|}{\texttt{OSCCON}}\\
  \hline
  % Ici on met le contenu des bits
  0 & 1 & 0 & 1 & 0 & 0 & 1 & 0\\
  \hline
  % Et ici les explications
  X 
  & \multicolumn{3}{>{\hsize=3\hsize}C|}{Horloge à 4Mhz} 
  & X & X 
  & \multicolumn{2}{>{\hsize=2\hsize}C|}{Utilisation de l'horloge interne}\\
  \hline
\end{tabularx}\\

Et on règle les interruptions:\\

\noindent
\begin{tabularx}{\textwidth}{|C|C|c|C|c|c|c|c|}
  \hline
  \multicolumn{8}{|>{\hsize=8\hsize}c|}{\texttt{INTCON}}\\
  \hline
  % Ici on met le contenu des bits
  1 & 1 & X & 1 & X & X & X & X\\
  \hline
  % Et ici les explications
  Activation des interruptions globales 
  & Activation des interruptions périphériques
  & X
  & Activation de l'interruption sur INT0
  & X & X & X & X\\
  \hline
\end{tabularx}\\\\

\noindent
\begin{tabularx}{\textwidth}{|C|C|c|c|c|c|c|c|}
  \hline
  \multicolumn{8}{|>{\hsize=8\hsize}c|}{\texttt{INTCON2}}\\
  \hline
  % Ici on met le contenu des bits
  1 & 0 & X & X & X & X & X & X\\
  \hline
  % Et ici les explications
  Interruption sur front descendant
  & \textit{À compléter}
  & X & X & X & X & X & X\\
  \hline
\end{tabularx}\\

\section{Application de la tare}

Une tare peut être prise lors de l'appui sur le bouton. Celui-ci est relié à une pin capable de déclencher une interruption matérielle pour exécuter une routine.\\

\begin{lstlisting}
HIGH_ISR
    ; Reinitialise le bit dinterruption
    BCF INTCON, 1
    CALL TARE
    RETFIE  FAST
\end{lstlisting}

La routine de tare prend une mesure de l'entrée analogique et enregistre la valeur brute dans un registre mémoire.\\

\begin{lstlisting}
TARE
    CALL ACQUISITION
    MOVFF RESULTLO, DEAD_WEIGHT
    RETURN
\end{lstlisting}

Cette valeur sera soustraite à la valeur mesurée avant conversion et affichage du poids.\\

\begin{lstlisting}
SHOWACQ
    CALL CLEARDISPLAY
    CALL ACQUISITION      ; Recuperation du poids (valeur brute)
    MOVF DEAD_WEIGHT, 0   
    SUBWF RESULTLO, 1     ; Application de la tare
\end{lstlisting}

\section{Application d'un coefficient correcteur}

La tension de sortie de la cellule de pesée est linéaire. À l'aide de différents poids, nous avons mesuré et placé sur un graphe les valeurs retournées par le CAN. À ce graphe, nous avons rajouté les valeurs que nous attendions (500 pour 500 grammes par exemple). On peut rapidement voir, à première vue, que la pente provenant du CAN doit être corrigée pour obtenir la même pente que le poids réel. Pour approximer la pente du CAN, nous avons calculé les pentes entre chaque segments puis nous avons fait la moyenne de ceux-ci. Nous avons supprimé une partie des dernières mesures, celles-ci paraissant fausses.\\

\begin{tikzpicture}
\begin{axis}
\addplot table [x=Weight, y=CAN output, col sep=comma] {../values.csv};
\addplot table [x=Weight, y=Weight, col sep=comma] {../values.csv};
\end{axis}
\end{tikzpicture}

Une fois la pente trouvée, nous avons pu calculer un coefficient correcteur à appliquer sur la valeur du CAN après suppression de la tare. Pour vérifier notre coefficient, nous avons calculé les valeurs corrigées. On peut observer une différence de maximum $\pm{2}$ grammes.\\

Comme nous ne pouvons pas multiplier par un nombre décimal en assembleur, nous devons multiplier à l'aide d'une fraction qui s'approche le plus possible du coefficient correcteur. En assembleur, pour multiplier par une fraction, il faut multiplier par le numérateur puis faire une division euclidienne par le dénominateur. Comme on multiplie par le numérateur, il faut veiller à ne pas utiliser de valeurs trop élevées pour que la valeur maximum ne déborde pas au-delà des 2 octets disponibles.\\

Nous avons calculé un coefficient correcteur de 1,07954. En arrondissant à 1,08, nous pouvons effectuer la conversion à l'aide d'une fraction $\frac{27}{25}$, ce qui nous permet de rester dans la gamme des valeurs disponibles ($1023 * 27 = 27621$)\\

\section{Problèmes rencontrés}

Lors du premier branchement de la cellule de pesée sur l'oscilloscope, nous n'arrivions pas à observer un changement de tension significatif. Nous avons donc rapidement monté la cellule de pesée sur un support permettant de la manipuler sans la maintenir gauchement sur un bord de table.\\

Une fois la cellule montée sur le support, nous avons pu aisément observer la variation de tension en fonction de la variation de poids et nous avons pu commencer à calculer le gain nécessaire.

\end{document}
